\chapter{Conclusion and Future Scope}
\section{Conclusion}
This project effort gives several indices for security analysis contingency screening. Again after disturbance has been addressed, these indicators are focused on coherency, transient energy conversion between kinetic and potential energy, dot products, and generator coupling parameters. These indices are quick to calculate because they don't require a lengthy simulation to determine if a situation is stable or unstable. They are precise because they have a high probability of capturing all unstable circumstances. These indices have been examined on a variety of test systems in the background while developing project including the discussed contingencies, and the findings indicate that the result provided by each of the indicator is correct and match to the original data. The possibility of mis-detection is zero till now, so this could be useful in future contingency screening.

 \pagebreak

\section{Future Scope}


The Catastrophic indicators have been analysed through a single method i.e. observing sudden rise in the curve to analyse the system behaviour in my paper. But there is need to define Threshold values for each indicator to observe the system behaviour after fault. Finding Threshold values through any type of optimization method or any other makes this Indicator Evaluation a complete process. In \cite{8}, the age of the possibility dataset and the calculation of execution allots are communicated via MATLAB in both offline and web-based modes. High-performance computing (HPC) \cite{FR1} or equal handling engineering \cite{FR2}, which is one of the conceivable forthcoming bearings identified, can be used to construct an internet-based structure.

