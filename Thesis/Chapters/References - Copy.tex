\begin{thebibliography}{99}
\begin{singlespace}
% Introductory
\bibitem{SAPS} Shakerighadi, B., Peyghami, S., Ebrahimzadeh, E., Blaabjerg, F. and Leth Back, C., 2020. A new guideline for security assessment of power systems with a high penetration of wind turbines. \textit{Applied Sciences, 10}(9), p.3190.

\bibitem{I6} Mittal, A., Hazra, J., Jain, N., Goyal, V., Seetharam, D.P. and Sabharwal, Y., 2011, August. Real time contingency analysis for power grids. In \textit{European Conference on Parallel Processing }(pp. 303-315). Springer, Berlin, Heidelberg.



%Literature Survey
\bibitem{L1}  Zimmerman, R.D., Murillo-Sánchez, C.E. and Thomas, R.J., 2010. MATPOWER: Steady-state operations, planning, and analysis tools for power systems research and education. \textit{IEEE Transactions on power systems, 26}(1), pp.12-19.

\bibitem{L2} Yang, H., Niu, K., Xu, D. and Xu, S., 2021. Analysis of power system transient stability characteristics with the application of massive transient stability simulation data.\textit{ Energy Reports, 7}, pp.111-117.

\bibitem{L3} Genc, I., Diao, R. and Vittal, V., 2010, July. Computation of transient stability related security regions and generation rescheduling based on decision trees. In \textit{IEEE PES General Meeting} (pp. 1-6). IEEE.

\bibitem{L4} Wehenkel, L. and Pavella, M., 1993. Decision tree approach to power systems security assessment. \textit{International Journal of Electrical Power \& Energy Systems, 15}(1), pp.13-36.

\bibitem{L5} Wehenkel, L. and Pavella, M., 1991. Decision trees and transient stability of electric power systems. \textit{Automatica, 27}(1), pp.115-134.

\bibitem{L6} Zhang, Y., Liu, W., Huan, Y., Zhou, Q. and Wang, N., 2020. Online Dynamic Total Transfer Capability Estimation Using Cotraining-Style Semi-Supervised Regression. \textit{IEEE Access, 8}, pp.94054-94064.

\bibitem{L7} Jafarzadeh, S., Genc, I. and Nehorai, A., 2021. Real-time transient stability prediction and coherency identification in power systems using Koopman mode analysis. \textit{Electric Power Systems Research, 201}, p.107565.
\bibitem{L8} Teixeira, M.O., Melo, I.D. and João Filho, A.P., 2021. An optimisation model based approach for power systems voltage stability and harmonic analysis. \textit{Electric Power Systems Research, 199}, p.107462.

\bibitem{L9} Meegahapola, L.G., Bu, S., Wadduwage, D.P., Chung, C.Y. and Yu, X., 2020. Review on oscillatory stability in power grids with renewable energy sources: Monitoring, analysis, and control using synchrophasor technology. \textit{IEEE Transactions on Industrial Electronics, 68}(1), pp.519-531.

\bibitem{L10} Chompoobutrgool, Y., Vanfretti, L. and Ghandhari, M., 2011. Survey on power system stabilizers control and their prospective applications for power system damping using Synchrophasor‐based wide‐area systems. \textit{European Transactions on Electrical Power, 21}(8), pp.2098-2111.

\bibitem{L11} Ghaedi, S., Abazari, S. and Markadeh, G.A., 2021. Transient stability improvement of power system with UPFC control by using transient energy function and sliding mode observer based on locally measurable information. \textit{Measurement, 183}, p.109842.

\bibitem{L12} Sobbouhi, A.R. and Vahedi, A., 2021. Transient stability prediction of power system; a review on methods, classification and considerations. \textit{Electric Power Systems Research, 190}, p.106853.

\bibitem{L13} Kamwa, I., Samantaray, S.R. and Joos, G., 2010. Catastrophe predictors from ensemble decision-tree learning of wide-area severity indices. \textit{IEEE Transactions on Smart Grid, 1}(2), pp.144-158.

\bibitem{L15} Villa-Acevedo, W.M., López-Lezama, J.M., Colomé, D.G. and Cepeda, J., 2022. Long-term voltage stability monitoring of power system areas using a kernel extreme learnin\textit{g machine approach. Alexandria Engineering Journal, 61}(2), pp.1353-1367.

% Project Motivation

\bibitem{1} Abhyankar, S., Geng, G., Anitescu, M., Wang, X. and Dinavahi, V., 2017. Solution techniques for transient stability-constrained optimal power flow–Part I. \textit{IET Generation, Transmission \& Distribution, 11}(12), pp.3177-3185.

\bibitem{2} Geng, G., Abhyankar, S., Wang, X. and Dinavahi, V., 2017. Solution techniques for transient stability-constrained optimal power flow–Part II. \textit{IET Generation, Transmission \& Distribution, 11}(12), pp.3186-3193.




\bibitem{5} Sobbouhi, A.R. and Vahedi, A., 2020. Online synchronous generator out-of-step prediction by ellipse fitting on acceleration power–Speed deviation curve. \textit{International Journal of Electrical Power \& Energy Systems, 119}, p.105965.

\bibitem{I5} Sobbouhi, A.R. and Vahedi, A., 2021. Transient stability improvement based on out-of-step prediction. \textit{Electric Power Systems Research, 194}, p.107108.

\bibitem{I3} Sobbouhi, A.R. and Vahedi, A., 2021. Transient stability prediction of power system; a review on methods, classification and considerations. \textit{Electric Power Systems Research, 190}, p.106853.

%

%
\bibitem{7} Balu, N.J., Lauby, M.G. and Kundur, P., 1994. Power system stability and control. \textit{Electrical Power Research Institute, McGraw-Hill Professional, 7}

\bibitem{8} Ravikumar, G. and Khaparde, S.A., 2016. Taxonomy of PMU data based catastrophic indicators for power system stability assessment. \textit{IEEE Systems Journal, 12}(1), pp.452-464.

\bibitem{9} Hashim, H., Abidin, I., Rashid, H., Zulkapli, N., Ibrahim, S., Kamar, S. and Sofizan, N., 2013. An out-of-step detection based on transient stability index. \textit{Aust. J. Basic Appl. Sci., 7}(4), pp.353-365.
%
\bibitem{EPM} Fu, C. and Bose, A., 1999. Contingency ranking based on severity indices in dynamic security analysis. \textit{IEEE Transactions on power systems, 14}(3), pp.980-985.
%
\bibitem{CSA} Kamwa, I., Grondin, R. and Loud, L., 2001. Time-varying contingency screening for dynamic security assessment using intelligent-systems techniques. \textit{IEEE Transactions on Power Systems, 16}(3), pp.526-536.
%
\bibitem{CSA_WASI} Kamwa, I., Samantaray, S.R. and Joos, G., 2009. Development of rule-based classifiers for rapid stability assessment of wide-area post-disturbance records. \textit{IEEE Transactions on Power Systems, 24}(1), pp.258-270.

\bibitem{ISPI} Gomez, O. and Rios, M.A., 2012, January. Inter-area stability prediction index based on phasorial measurement. In \textit{2012 IEEE PES Innovative Smart Grid Technologies (ISGT) }(pp. 1-8). IEEE.
%

\bibitem{MattransSoft} Ravikumar, G., 2016. MATTRANS: a MATLAB power system transient stability simulation package. \textit{url: https://github. com/gelliravi/MatTrans.}
%
\bibitem{FR1} Green, R.C., Wang, L. and Alam, M., 2011, July. High performance computing for electric power systems: Applications and trends. In \textit{2011 IEEE Power and Energy Society general meeting} (pp. 1-8). IEEE.
%
\bibitem{FR2} Jalili-Marandi, V., Zhou, Z. and Dinavahi, V., 2012, July. Large-scale transient stability simulation of electrical power systems on parallel GPUs. In \textit{2012 IEEE Power and Energy Society General Meeting }(pp. 1-11). IEEE.
%
%
%
%\bibitem{}
%
%\bibitem{}
%
%\bibitem{}
%
%\bibitem{}
%
%\bibitem{}
%
%\bibitem{}
%
%\bibitem{}
%
%\bibitem{}
%
%\bibitem{}
%
%\bibitem{}
%
%\bibitem{}
%
%\bibitem{}
%
%\bibitem{}
\end{singlespace}

\end{thebibliography}